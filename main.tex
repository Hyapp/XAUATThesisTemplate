\documentclass[doublesided, ctexart]{Style/xauat}
\usepackage{booktabs}
\usepackage[left=2.5cm, right=2.5cm, bottom=2cm, top=4.5cm, footskip=1cm, bindingoffset=0.5cm]{geometry}
\usepackage{amscd,amssymb,amsfonts,amsbsy,amsmath,verbatim,color}
\usepackage[super,myhdr,list]{Style/commons}% common settings
\usepackage{Style/custom}% user defined commands
\captionsetup{labelformat=default,labelsep=quad} %去除冒号
\renewcommand {\theequation} {\thechapter{}-\arabic{equation}}
\usepackage{array, caption, threeparttable}
\captionsetup[table]{singlelinecheck=false}%%%% -------------------------------------------------------------------------------%%%%******************************** Content *****************************************
\usepackage{chngpage}
\usepackage{mathrsfs}
\usepackage{float}

\begin{document}

%%%%******************************** Frontmatter *************************************
%% Frontmatter of Title page, Table of contents, Preface chapter.
\frontmatter

% title page

%%
%%% Generate English Title
%%
%\makeenglishtitle{}
%%
%%% >>> Author's declaration
%%
%\makedeclaration{}
%%
%%% >>> Abstract
%%
%%% >>> Title Pa
%%
%%% Chinese Title Page
%%
\schoollogo{scale=0.6}{Imgs/xiaoming.png}% university logo

\author{张三}% name of author
\confidential{888888888}% show confidential tag  学号
\school{未来学院}
\major{未来} %专业
\advisor{王五\quad 教授}% 指导教师
\defensedate{2024~年~1~月~1~日}

\title[西安建筑科技大学硕士学位论文]{西安建筑科技大学建校研究}% \title[short title for headers]{Long title of thesis}


\degreetype{}% degree type

\institute{\heiti 西安建筑科技大学}% institute of author
%\chinesedate{2014~年~06~月}% only need for user customized date
%%
%%% English Title Page
%%
\englishtitle{What is the meaning of life, the universe and everything?}
\englishauthor{Javascript, Huang}
\englishadvisor{Prof. Wu Wang}
\englishmajor{Science}
\englishdate{June 3rd, 2025}
\englishDegreeType{Acadamic Degree}
%%
%%% Generate Chinese Title
%%
\maketitle
%%
%%% Generate English Title
%%
\makeenglishtitle
%%
\makecommittee{}

%%% >>> Author's declaration
%%
\makedeclaration{}
%%
%%% >>> Abstract
%%


%abstract, dedication
\pagenumbering{Roman}
% !Mode:: "TeX:UTF-8"
\fancyhead[C]{\leftmark}
\fancyhead[CO]{\leftmark} % 在偶数页C显示当前章的标题
\fancyhead[CE]{西安建筑科技大学硕士学位论文} % 在奇数页C显示学校

\begin{abstract}
错里错以错劝哥哥、情中情因情感妹妹
\end{abstract}

\begin{keywords}
大,小,方,元
\end{keywords}

\vspace{\baselineskip} % 段前空一行
\noindent % 取消首行缩进
\textbf{\fontsize{12}{14}\selectfont 论文类型: }神学研究

\vspace{1.5\baselineskip} % 段前0.5行
\noindent % 取消首行缩进
本研究得到国家自然科学基金(编号:888888888)资助.
\cleardoublepage


\fancyhead[C]{\leftmark}
\fancyhead[CO]{\leftmark} % 在偶数页C显示当前章的标题
\fancyhead[CE]{西安建筑科技大学硕士学位论文} % 在奇数页C显示学校

\begin{englishabstract}
English Abstract...
\end{englishabstract}

\begin{englishkeywords}
Big, Small, Square, Circle
\end{englishkeywords}

\vspace{\baselineskip} % 段前空一行
\noindent % 取消首行缩进
\textbf{\fontsize{12}{14}\selectfont Type of Dissertation: }Theology!!! Research.

\vspace{1.5\baselineskip} % 段前0.5行
\noindent % 取消首行缩进
This research is supported by the National Natural Science Foundation of China under Grant (888888888).
\cleardoublepage



\pagestyle{fancy}
\fancyhead{} % 清除所有页眉内容

\fancyhead[C]{\目录}
\fancyhead[CO]{\leftmark} % 在偶数页C显示当前章的标题
\fancyhead[CE]{西安建筑科技大学硕士学位论文} % 在奇数页C显示学校

\tableofcontents % contents catalog%目录

%%
%% >>> Nomenclatures
%%
\chapter{主要符号表}
\nomenclatureitem[\textbf{符号}]{\textbf{符号含义}}
\nomenclatureitem[$p$]{P}

%%符号表

%%%%% --------------------------------------------------------------------------------
%%%%******************************** Mainmatter **************************************
%% Main topics.
\mainmatter
%%% >>> Main Contents%正文
\chapter{引言}
\section{研究背景和意义}
\section{国内外研究现状}
\section{本文主要工作及内容安排}

这是一个测试用的句子
\chapter{使用方法演示;大章标题}

\section{节}

\begin{axiom}
    $\sin^2 x + \cos^2 x = 1$
\end{axiom}

\subsection{小节}

\textcolor{red}{下面命题演示了如何写非编号公式,编号公式与引用编号公式!}
\begin{proposition} 正弦定理
    \[
        \frac{a}{\sin A} = \frac{b}{\sin B} = \frac{c}{\sin C}
    \]

    \begin{proof}
        在直角三角形中,利用勾股定理 $a^2 + b^2 = c^2$。
        而直角对的边长为 $c$,即有 $\sin C = 1$,进而 $\frac{c}{\sin C} = c$。
        此时 $\sin A = \frac{a}{c}$,因此 $\frac{a}{\sin A} = \frac{a}{\frac{a}{c}} = c$。
        类似地,$\frac{b}{\sin B} = c$。
        这就对直角三角形证明了正弦定理。

        对于任意三角形,三个角中必有最大的角设为 $A$,以此角对应的顶点向对应的边 $BC$ 作垂线,记新得到的点为 $D$,此时得到两个子直角三角形 $ABD$ 和 $ACD$。
        
        根据直角三角形的正弦定理分别有
        \begin{equation}\label{formula 1}
            c = \frac{c}{\sin \frac{\pi}{2}} = \frac{AD}{\sin B} = \frac{BD}{\sin BAD}
        \end{equation}
        和
        \begin{equation}\label{formula 2}
            b = \frac{b}{\sin \frac{\pi}{2}} = \frac{AD}{\sin C} = \frac{DC}{\sin DAC}
        \end{equation}
        
        根据式 \ref{formula 1} 和 \ref{formula 2} 有 $\sin C = \frac{AD}{b}$ 和 $\sin B = \frac{AD}{c}$。
        \[
            \frac{c}{\sin C} = \frac{c}{\frac{AD}{b}} = \frac{bc}{AD}
            \text{ 和 }
            \frac{b}{\sin B} = \frac{b}{\frac{AD}{c}} = \frac{bc}{AD}
        \]
        这就证明了 $\frac{b}{\sin B} = \frac{c}{\sin C}$。

        再次对角 $B$ 使用上述方法,可以得到 $\frac{a}{\sin A} = \frac{c}{\sin C}$。
        这就证明了正弦定理。
    \end{proof}
\end{proposition}

\textcolor{red}{下面例演示了如何使用多行公式,它只产生一个居中编号,给定理等定义名字,后面将引用}
\begin{example}\label{exap: sin cos infty}
    \begin{equation}
        \begin{aligned}
            \lim_{x\to +\infty} \frac{\sin x}{x} = 0 \\
            \lim_{x\to +\infty} \frac{\cos x}{x} = 0
        \end{aligned}
    \end{equation}
\end{example}

\textcolor{red}{下面推论的证明中引用了上一个例子}
\begin{corollary}
    \[
        \lim_{x\to +\infty} \frac{\sin^2 x}{x^2} = 0
    \]

    \begin{proof}
        根据例 \ref{exap: sin cos infty},显然成立。
    \end{proof}
\end{corollary}

\textcolor{red}{以下注演示了如果定理等和证明若起始没有内容,或是大公式,则应当在后面加入空格 $\backslash$ quad 来 让其自动对齐,否则将产生不正确的上下对齐}
\begin{remark} \quad
    \[
        \lim_{x\to 0} \frac{x}{x} = 1
    \]
\end{remark}

\begin{equation}
    \frac{\mathrm{d}}{\mathrm{d}x} \tan x = \sec^2 x
\end{equation}

\textcolor{red}{下面例演示了插入图片并缩放为0.5倍原始大小,建议将图片全放入 Imgs 文件夹中}

\begin{figure}[h]
    \centering % 图片居中
    \includegraphics[scale=0.5]{Imgs/xiaoming.png}
    \caption{这是校标志}
\end{figure}

\textcolor{red}{下面段落演示了如何以不同方式引用参考文献}

这个是测试引用的段落\cite{Ibrahim2022},具体证明见 \cite[定理 8]{SB2077}。
数学使人迷惑\supercite{SB2077}。

\begin{theorem}
    这是第一行定理
    这是第一行定理
    这是第一行定理
    这是第一行定理
    这是第一行定理
    这是第一行定理
    这是第一行定理
    这是第一行定理
    \[
        x + x + x + x +
        x + x + x + x +
        x + x + x + x +
        x + x + x + x
    \]
\end{theorem}

\begin{theorem}
    这是第二行定理
\end{theorem}





%%% 参考文献
% 如果想恢复之前的方案,请去掉方式1后面的\input 前的 %, 并且在方式2中的 \bibliography{...} 加上 % 或删去这一行!
\backmatter
\intotoc{\bibname}

% 方式1: 手动写进 Texs/References.tex 文件中
%\pagestyle{fancy}
\fancyhead{} % 清除所有页眉内容

\fancyhead[C]{\目录}
\fancyhead[CO]{\leftmark} % 在偶数页C显示当前章的标题
\fancyhead[CE]{西安建筑科技大学硕士学位论文} % 在奇数页C显示学校
%{\centering\song\begin{thebibliography}{99}}
\begin{thebibliography}{99}

\bibitem{Howie1995} Howie. Fundamentals of Semigroups Theory[B]. Oxford Press.

\end{thebibliography}

 % 在 Texs/Referebces.tex 中手写参考文献

% 方式2: 使用多阶段编译法用 .bib 文件中编译
% 引用 自动生成或手写的 Bibilo/Biblio.bib 中的参考文献
% Biblio.bib 文件中的条目可以在网站上的引用中选 bib 格式,复制粘贴到 Biblio.bib 中即可。
\bibliography{Biblio/Bibliography.bib} 

% 引用方式:
% 1. 在定理、定义等后面使用 \themcite{...}
% 2. 在正文中使用 \cite{...}

%%% >>> Other contents
%%
%%% >>> Resume and Published papers
%%

%%
%%% >>> Acknowledgements
%%

\chapter{致 \quad 谢}

\quad 感谢导师






 % 致谢
\pagestyle{fancy}
\fancyhead{} % 清除所有页眉内容

\fancyhead[C]{\目录}
\fancyhead[CO]{\leftmark} % 在偶数页C显示当前章的标题
\fancyhead[CE]{西安建筑科技大学硕士学位论文} % 在奇数页C显示学校
\chapter{攻读硕士学位期间取得的研究成果}

\renewcommand{\labelenumi}{[\theenumi]}

{\noindent\textbf{完成的学术论文}}
\begin{enumerate}\setlength{\itemsep}{0pt}
\item 王五, \textbf{张三}. 一类大学的研究[J]. 科学, 2077.
\end{enumerate}

{\noindent\textbf{科研项目及获奖}}
\begin{enumerate}\setlength{\itemsep}{0pt}
\item “中国光谷·华为杯”第十九届中国研究生数学建模竞赛一等奖, 2025年1月.
\end{enumerate}
 % 科研成果
\end{document}







