\chapter{使用方法演示;大章标题}

\section{节}

\begin{axiom}
    $\sin^2 x + \cos^2 x = 1$
\end{axiom}

\subsection{小节}

\textcolor{red}{下面命题演示了如何写非编号公式,编号公式与引用编号公式!}
\begin{proposition} 正弦定理
    \[
        \frac{a}{\sin A} = \frac{b}{\sin B} = \frac{c}{\sin C}
    \]

    \begin{proof}
        在直角三角形中,利用勾股定理 $a^2 + b^2 = c^2$。
        而直角对的边长为 $c$,即有 $\sin C = 1$,进而 $\frac{c}{\sin C} = c$。
        此时 $\sin A = \frac{a}{c}$,因此 $\frac{a}{\sin A} = \frac{a}{\frac{a}{c}} = c$。
        类似地,$\frac{b}{\sin B} = c$。
        这就对直角三角形证明了正弦定理。

        对于任意三角形,三个角中必有最大的角设为 $A$,以此角对应的顶点向对应的边 $BC$ 作垂线,记新得到的点为 $D$,此时得到两个子直角三角形 $ABD$ 和 $ACD$。
        
        根据直角三角形的正弦定理分别有
        \begin{equation}\label{formula 1}
            c = \frac{c}{\sin \frac{\pi}{2}} = \frac{AD}{\sin B} = \frac{BD}{\sin BAD}
        \end{equation}
        和
        \begin{equation}\label{formula 2}
            b = \frac{b}{\sin \frac{\pi}{2}} = \frac{AD}{\sin C} = \frac{DC}{\sin DAC}
        \end{equation}
        
        根据式 \ref{formula 1} 和 \ref{formula 2} 有 $\sin C = \frac{AD}{b}$ 和 $\sin B = \frac{AD}{c}$。
        \[
            \frac{c}{\sin C} = \frac{c}{\frac{AD}{b}} = \frac{bc}{AD}
            \text{ 和 }
            \frac{b}{\sin B} = \frac{b}{\frac{AD}{c}} = \frac{bc}{AD}
        \]
        这就证明了 $\frac{b}{\sin B} = \frac{c}{\sin C}$。

        再次对角 $B$ 使用上述方法,可以得到 $\frac{a}{\sin A} = \frac{c}{\sin C}$。
        这就证明了正弦定理。
    \end{proof}
\end{proposition}

\textcolor{red}{下面例演示了如何使用多行公式,它只产生一个居中编号,给定理等定义名字,后面将引用}
\begin{example}\label{exap: sin cos infty}
    \begin{equation}
        \begin{aligned}
            \lim_{x\to +\infty} \frac{\sin x}{x} = 0 \\
            \lim_{x\to +\infty} \frac{\cos x}{x} = 0
        \end{aligned}
    \end{equation}
\end{example}

\textcolor{red}{下面推论的证明中引用了上一个例子}
\begin{corollary}
    \[
        \lim_{x\to +\infty} \frac{\sin^2 x}{x^2} = 0
    \]

    \begin{proof}
        根据例 \ref{exap: sin cos infty},显然成立。
    \end{proof}
\end{corollary}

\textcolor{red}{以下注演示了如果定理等和证明若起始没有内容,或是大公式,则应当在后面加入空格 $\backslash$ quad 来 让其自动对齐,否则将产生不正确的上下对齐}
\begin{remark} \quad
    \[
        \lim_{x\to 0} \frac{x}{x} = 1
    \]
\end{remark}

\begin{equation}
    \frac{\mathrm{d}}{\mathrm{d}x} \tan x = \sec^2 x
\end{equation}

\textcolor{red}{下面例演示了插入图片并缩放为0.5倍原始大小,建议将图片全放入 Imgs 文件夹中}

\begin{figure}[h]
    \centering % 图片居中
    \includegraphics[scale=0.5]{Imgs/xiaoming.png}
    \caption{这是校标志}
\end{figure}

\textcolor{red}{下面段落演示了如何以不同方式引用参考文献}

这个是测试引用的段落\cite{Ibrahim2022},具体证明见 \cite[定理 8]{SB2077}。
数学使人迷惑\supercite{SB2077}。

\begin{theorem}
    这是第一行定理
    这是第一行定理
    这是第一行定理
    这是第一行定理
    这是第一行定理
    这是第一行定理
    这是第一行定理
    这是第一行定理
    \[
        x + x + x + x +
        x + x + x + x +
        x + x + x + x +
        x + x + x + x
    \]
\end{theorem}

\begin{theorem}
    这是第二行定理
\end{theorem}


