\chapter{预备知识}

\section{基本概念}

\begin{definition}\themcite{Howie1995}
    令$S$为非空集合, $S$上有一个二元运算 $\mu:S\times S \to S$, 并且运算 $\mu$ 满足结合律:$\forall a,b,c \in S$, 满足 $((a,b)\mu, c)\mu = (a, (b, c)\mu)\mu$, 则称集合$S$和二元运算$\mu$为半群。
\end{definition}

通常为了简便起见,记二元运算$\mu$为乘法,即记$(a,b)\mu$,为$a\cdot b$,或进一步简记为$ab$。

\begin{definition}\themcite{Howie1995}
    令$S$为半群,$T$为$S$的一个子集,如果$T$关于$S$上的二元运算$\mu$封闭,即 $\forall a,b \in T$, 有$ab\in T$, 则称 $T$ 连同$S$上二元运算$\mu$ 在 $T$ 上的限制称为$S$ 的子半群。
\end{definition}

\begin{definition}\themcite{Howie1995}
    令$S$为半群,如果存在元素$1 \in S$, 并且满足$\forall s\in S, 1s=s1=s$ ,则称元素 $1$ 为半群$S$的幺元。
\end{definition}

\begin{lemma}\themcite{Howie1995}
    令$S$为半群,如果$S$有幺元,那么它只有一个幺元。
\end{lemma}

\begin{definition}\themcite{Howie1995}
    如果半群$S$中含有幺元,则称其为幺半群。
\end{definition}

\begin{definition}\themcite{Howie1995}
    令$S$是半群,如果$S$不含幺元,则添加一个幺元$1$,并且任意$s\in S$有 $1s = s1 = s$. 那么$S\cup{1}$是幺半群,记为$S^1$。即$S^1$ 表示如下半群
    \begin{equation*}
        S^1=\begin{cases}
            S & \text{若}S \text{包含幺元} \\
            S \cup {1} & \text{若}S \text{不含幺元}
        \end{cases}
    \end{equation*}
\end{definition}

\begin{definition}\themcite{Howie1995}
    令$S$是半群,如果元素$a\in S$,满足 $aa=a$,则称$a$为幂等元。
\end{definition}

\begin{definition}
    若半群$S$中的元素都是幂等元,则称半群$S$为带。

    (i) 若对于带$S$中任意两个元素的二元运算都可交换,即 $ab=ba$,则称带$S$为半格。
\end{definition}

\begin{definition}\themcite{Howie1995}
    令 $S$ 为半群,$\rho$ 为 $S$ 上的一个二元关系,并且 $S$ 中的元素 $a, b$ 有$\rho$ 关系

    (i) 对于任意 $c \in S$, 有 $ca \rho cb$,则称 $\rho$ 是左相容的;

    (ii) 对于任意 $c \in S$, 有 $ac \rho bc$,则称 $\rho$ 是右相容的;

    (iii) 若$\rho$ 即是左相容的,也是右相容的,则称 $\rho$ 是相容的。
\end{definition}

\begin{definition}\themcite{Howie1995}
    令 $S$ 为半群,$\rho$ 为 $S$ 上的一个等价关系。若

    (i) $\rho$ 是左相容的,则称 $\rho$ 是左同余;

    (ii) $\rho$ 是右相容的,则称 $\rho$ 是右同余;

    (iii) $\rho$ 是相容的,则称 $\rho$ 是同余。
\end{definition}




\section{格林关系与正则半群}
Green于1965年在半群上定义了格林关系,开启了研究半群的工具。
\begin{definition}\themcite{Howie1995}
    令$S$为半群,$a, b$ 为$S$的两个元素。

    (i) 若存在元素$x, y$ 使得 $xa=b$ 并且 $yb=a$,则称$a\mathcal{L}b$。

    (ii) 若存在元素 $x, y$ 使得 $ax=b$ 并且 $by=a$,则称$a\mathcal{R}b$。

    (iii) 若 $a\mathcal{L}b$ 并且 $a\mathcal{R}b$,则称$a\mathcal{H}b$。
    
    
\end{definition}
\section{广义格林关系与广义正则半群}
Pastijn在1975年弱化了Green关系的条件,形成了 $\ast$-Green 关系。

\begin{definition}
    令 $S$ 为半群,$a, b$ 为 $S$ 的两个元素。

    (i) $\mathcal{L}^{\ast} = \{(a,b)\in S\times S| \forall x,y \in S^1 \rightarrow ax=ay \Leftrightarrow bx=by\}$;

    (ii) $\mathcal{R}^{\ast} = \{(a,b)\in S\times S| \forall x,y \in S^1 \rightarrow xa=ya \Leftrightarrow xb=yb\}$;

    (iii) $\mathcal{H}^{\ast} = \mathcal{L}^{\ast} \cap \mathcal{R}^{\ast}$
\end{definition}

Lawson在1990年进一步弱化了Pastijn提出的 $\ast$-Green关系,形成了$\sim$-Green关系。

\begin{definition}
    令 $S$ 为半群,$a, b$ 为 $S$ 的两个元素。

    (i) $\tilde{\mathcal{L}} = \{(a,b)\in S\times S| \forall e \in E \rightarrow ae=a \Leftrightarrow be=b\}$;

    (ii) $\tilde{\mathcal{R}} = \{(a,b)\in S\times S| \forall e \in E \rightarrow ea=a \Leftrightarrow eb=b\}$;

    (iii) $\tilde{\mathcal{H}} = \tilde{\mathcal{L}} \cap \tilde{\mathcal{R}}$
\end{definition}

\begin{axiom}
    $\sin^2 x + \cos^2 x = 1$
\end{axiom}

\begin{proposition} 正弦定理
    \[
        \frac{a}{\sin A} = \frac{b}{\sin B} = \frac{c}{\sin C}
    \]
\end{proposition}

\begin{example} 
    \[
        \lim_{x\to +\infty} \frac{\sin x}{x} = 0
    \]
\end{example}

\begin{corollary}
    \[
        \lim_{x\to +\infty} \frac{\sin^2 x}{x^2} = 0
    \]
\end{corollary}

\begin{remark}
    \[
        \lim_{x\to 0} \frac{x}{x} = 1
    \]
\end{remark}


这个是测试引用的段落\cite{Ibrahim2022},你觉得呢\cite[定理 8]{SB2077}。

冯唐易老,李广难封\supercite{SB2077}。


