\documentclass[doublesided]{Style/xauat}
\usepackage[super,myhdr,list]{Style/commons}% common settings
\usepackage{Style/custom}% user defined commands

\captionsetup[table]{singlelinecheck=false}%%%% -------------------------------------------------------------------------------%%%%******************************** Content *****************************************

\begin{document}

%%%%******************************** Frontmatter *************************************
%% Frontmatter of Title page, Table of contents, Preface chapter.
\frontmatter


%%
%%% Generate English Title
%%
%\makeenglishtitle{}
%%
%%% >>> Author's declaration
%%
%\makedeclaration{}
%%
%%% >>> Abstract
%%
%%% >>> Title Pa
%%
%%% Chinese Title Page
%%
\schoollogo{scale=0.6}{Imgs/xiaoming.png}% university logo

\author{张三}% name of author
\confidential{888888888}% show confidential tag  学号
\school{未来学院}
\major{未来} %专业
\advisor{王五\quad 教授}% 指导教师
\defensedate{2024~年~1~月~1~日}

\title[XXX大学硕士学位论文]{西安建筑科技大学建校研究}% \title[short title for headers]{Long title of thesis}


\degreetype{}% degree type

\institute{\heiti 西安建筑科技大学}% institute of author
%\chinesedate{2014~年~06~月}% only need for user customized date
%%
%%% English Title Page
%%
\englishtitle{What is the meaning of life, the universe and everything?}
\englishauthor{Javascript, Huang}
\englishadvisor{Prof. Wu Wang}
\englishmajor{Science}
\englishdate{June 3rd, 2025}
\englishDegreeType{Acadamic Degree}



% 答辩委员会信息,可以为空,但不能删!

\chiefName{主席名} 
\chiefTitle{主席职称} 
\chiefOrganisation{主席工作单位}

\staffNameOne{委员1}
\staffTitleOne{职务1}
\staffOrganisationOne{单位1}

\staffNameTwo{委员2}
\staffTitleTwo{职务2}
\staffOrganisationTwo{单位2}

\staffNameThree{委员3}
\staffTitleThree{职务3}
\staffOrganisationThree{单位3}

\staffNameFour{委员4}
\staffTitleFour{职务4}
\staffOrganisationFour{单位4}

\staffNameFive{委员5}
\staffTitleFive{职务5}
\staffOrganisationFive{单位5}

\staffNameSix{}
\staffTitleSix{}
\staffOrganisationSix{}

%%
%%% Generate Chinese Title
%%
% \maketitle
%%
%%% Generate English Title
%%
% \makeenglishtitle
%%
% \makecommittee

%%% >>> Author's declaration
%%
% \makedeclaration
%%
%%% >>> Abstract
%%
 % title page

%abstract, dedication
\pagenumbering{Roman}

% !Mode:: "TeX:UTF-8"
\fancyhead[C]{\leftmark}
\fancyhead[CO]{\leftmark} % 在偶数页C显示当前章的标题
\fancyhead[CE]{XXX大学硕士学位论文} % 在奇数页C显示学校

\begin{abstract}
错里错以错劝哥哥、情中情因情感妹妹
\end{abstract}

\begin{keywords}
大,小,方,元
\end{keywords}

\vspace{\baselineskip} % 段前空一行
\noindent % 取消首行缩进
{\heiti{论文类型:}}理论研究神学研究

\vspace{1.5\baselineskip} % 段前0.5行
\noindent % 取消首行缩进
本研究得到国家自然科学基金(编号:888888888)资助.
\cleardoublepage

\begin{englishabstract}
English Abstract...
\end{englishabstract}

\begin{englishkeywords}
Big, Small, Square, Circle
\end{englishkeywords}

\vspace{\baselineskip} % 段前空一行
\noindent % 取消首行缩进
\textbf{Type of Dissertation: }Theology!!! Research.

\vspace{1.5\baselineskip} % 段前0.5行
\noindent % 取消首行缩进
This research is supported by the National Natural Science Foundation of China under Grant (888888888).
\cleardoublepage



\pagestyle{fancy}

\tableofcontents % 目录

%%
%% >>> Nomenclatures
%%
\pagestyle{fancy}
\fancyhead{} % 清除所有页眉内容
\fancyhead[C]{\leftmark}
\fancyhead[CO]{\leftmark} % 在偶数页C显示当前章的标题
\fancyhead[CE]{西安建筑科技大学硕士学位论文} % 在奇数页C显示学校

\chapter{主要符号表}
\nomenclatureitem[\textbf{符号}]{\textbf{符号含义}}
\nomenclatureitem[$p$]{屁}

%%符号表

%%%%% --------------------------------------------------------------------------------
%%%%******************************** Mainmatter **************************************
%% Main topics.
\mainmatter
%%% >>> Main Contents%正文
\chapter{引言}
\section{研究背景和意义}
\section{国内外研究现状}
\section{本文主要工作及内容安排}

这是一个测试用的句子
\chapter{预备知识}

\section{基本概念}

\begin{definition}\themcite{Howie1995}
    令$S$为非空集合, $S$上有一个二元运算 $\mu:S\times S \to S$, 并且运算 $\mu$ 满足结合律:$\forall a,b,c \in S$, 满足 $((a,b)\mu, c)\mu = (a, (b, c)\mu)\mu$, 则称集合$S$和二元运算$\mu$为半群。
\end{definition}

通常为了简便起见,记二元运算$\mu$为乘法,即记$(a,b)\mu$,为$a\cdot b$,或进一步简记为$ab$。

\begin{definition}\themcite{Howie1995}
    令$S$为半群,$T$为$S$的一个子集,如果$T$关于$S$上的二元运算$\mu$封闭,即 $\forall a,b \in T$, 有$ab\in T$, 则称 $T$ 连同$S$上二元运算$\mu$ 在 $T$ 上的限制称为$S$ 的子半群。
\end{definition}

\begin{definition}\themcite{Howie1995}
    令$S$为半群,如果存在元素$1 \in S$, 并且满足$\forall s\in S, 1s=s1=s$ ,则称元素 $1$ 为半群$S$的幺元。
\end{definition}

\begin{lemma}\themcite{Howie1995}
    令$S$为半群,如果$S$有幺元,那么它只有一个幺元。
\end{lemma}

\begin{definition}\themcite{Howie1995}
    如果半群$S$中含有幺元,则称其为幺半群。
\end{definition}

\begin{definition}\themcite{Howie1995}
    令$S$是半群,如果$S$不含幺元,则添加一个幺元$1$,并且任意$s\in S$有 $1s = s1 = s$. 那么$S\cup{1}$是幺半群,记为$S^1$。即$S^1$ 表示如下半群
    \begin{equation*}
        S^1=\begin{cases}
            S & \text{若}S \text{包含幺元} \\
            S \cup {1} & \text{若}S \text{不含幺元}
        \end{cases}
    \end{equation*}
\end{definition}

\begin{definition}\themcite{Howie1995}
    令$S$是半群,如果元素$a\in S$,满足 $aa=a$,则称$a$为幂等元。
\end{definition}

\begin{definition}
    若半群$S$中的元素都是幂等元,则称半群$S$为带。

    (i) 若对于带$S$中任意两个元素的二元运算都可交换,即 $ab=ba$,则称带$S$为半格。
\end{definition}

\begin{definition}\themcite{Howie1995}
    令 $S$ 为半群,$\rho$ 为 $S$ 上的一个二元关系,并且 $S$ 中的元素 $a, b$ 有$\rho$ 关系

    (i) 对于任意 $c \in S$, 有 $ca \rho cb$,则称 $\rho$ 是左相容的;

    (ii) 对于任意 $c \in S$, 有 $ac \rho bc$,则称 $\rho$ 是右相容的;

    (iii) 若$\rho$ 即是左相容的,也是右相容的,则称 $\rho$ 是相容的。
\end{definition}

\begin{definition}\themcite{Howie1995}
    令 $S$ 为半群,$\rho$ 为 $S$ 上的一个等价关系。若

    (i) $\rho$ 是左相容的,则称 $\rho$ 是左同余;

    (ii) $\rho$ 是右相容的,则称 $\rho$ 是右同余;

    (iii) $\rho$ 是相容的,则称 $\rho$ 是同余。
\end{definition}




\section{格林关系与正则半群}
Green于1965年在半群上定义了格林关系,开启了研究半群的工具。
\begin{definition}\themcite{Howie1995}
    令$S$为半群,$a, b$ 为$S$的两个元素。

    (i) 若存在元素$x, y$ 使得 $xa=b$ 并且 $yb=a$,则称$a\mathcal{L}b$。

    (ii) 若存在元素 $x, y$ 使得 $ax=b$ 并且 $by=a$,则称$a\mathcal{R}b$。

    (iii) 若 $a\mathcal{L}b$ 并且 $a\mathcal{R}b$,则称$a\mathcal{H}b$。
    
    
\end{definition}
\section{广义格林关系与广义正则半群}
Pastijn在1975年弱化了Green关系的条件,形成了 $\ast$-Green 关系。

\begin{definition}
    令 $S$ 为半群,$a, b$ 为 $S$ 的两个元素。

    (i) $\mathcal{L}^{\ast} = \{(a,b)\in S\times S| \forall x,y \in S^1 \rightarrow ax=ay \Leftrightarrow bx=by\}$;

    (ii) $\mathcal{R}^{\ast} = \{(a,b)\in S\times S| \forall x,y \in S^1 \rightarrow xa=ya \Leftrightarrow xb=yb\}$;

    (iii) $\mathcal{H}^{\ast} = \mathcal{L}^{\ast} \cap \mathcal{R}^{\ast}$
\end{definition}

Lawson在1990年进一步弱化了Pastijn提出的 $\ast$-Green关系,形成了$\sim$-Green关系。

\begin{definition}
    令 $S$ 为半群,$a, b$ 为 $S$ 的两个元素。

    (i) $\tilde{\mathcal{L}} = \{(a,b)\in S\times S| \forall e \in E \rightarrow ae=a \Leftrightarrow be=b\}$;

    (ii) $\tilde{\mathcal{R}} = \{(a,b)\in S\times S| \forall e \in E \rightarrow ea=a \Leftrightarrow eb=b\}$;

    (iii) $\tilde{\mathcal{H}} = \tilde{\mathcal{L}} \cap \tilde{\mathcal{R}}$
\end{definition}



这个是测试引用的段落\cite{Ibrahim2022},你觉得呢\cite[定理 8]{SB2077}。

冯唐易老,李广难封\supercite{SB2077}。




%%% 参考文献
% 如果想恢复之前的方案,请去掉方式1后面的\input 前的 %, 并且在方式2中的 \bibliography{...} 加上 % 或删去这一行!
\backmatter
\intotoc{\bibname}

% ------------------------------------------------------
% 方式1: 手动写进 Texs/References.tex 文件中
%
\begin{thebibliography}{99}

% 在这个地方手写参考文献,格式与下面的 \bibitem{...}... 类似

\bibitem{Ibrahim2022}
Ibrahim~M~J, Sawudi~I~M, Imam~A~T.
On the Semigroup of Difunctional Binary Relations~[J].
FUDMA JOURNAL OF SCIENCES.
2022, 6~(4):  17--19.

\bibitem{ZS2077}
赵四, 王五.
关于未来的研究~[J].
自然.
2077, 1~(1):  1--1.

\end{thebibliography}

 % 在 Texs/Referebces.tex 中手写参考文献

% -------------------------------------------------------
% 方式2: 使用多阶段编译法用 .bib 文件中编译
% 编译过程为:
% XeLaTeX -> BibLaTeX -> XeLaTeX -> XeLaTeX
% Biblio.bib 文件中的条目可以在网站上的引用中选 bib 格式,
% 复制粘贴到 Biblio文件夹中的Bibliography.bib 中即可。
\bibliography{Biblio/Bibliography.bib} 

% 引用方式:见 README.pdf

%%% >>> Other contents
%%
%%% >>> Resume and Published papers
%%

%%
%%% >>> Acknowledgements
%%

\chapter{致 \quad 谢}

\quad 感谢导师






 % 致谢
\pagestyle{fancy}
\fancyhead{} % 清除所有页眉内容

\fancyhead[C]{\目录}
\fancyhead[CO]{\leftmark} % 在偶数页C显示当前章的标题
\fancyhead[CE]{西安建筑科技大学硕士学位论文} % 在奇数页C显示学校
\chapter{攻读硕士学位期间取得的研究成果}

{\noindent\textbf{完成的学术论文}}
\begin{enumerate}\setlength{\itemsep}{0pt}
\renewcommand{\labelenumi}{[\theenumi]}
\item 王五, \textbf{张三}. 一类大学的研究[J]. 科学, 2024.
\end{enumerate}

{\noindent\textbf{科研项目及获奖}}
\begin{enumerate}\setlength{\itemsep}{0pt}
\renewcommand{\labelenumi}{[\theenumi]}
\item “中国光谷·华为杯”第十九届中国研究生数学建模竞赛一等奖, 2025年1月.
\end{enumerate}
 % 科研成果
\end{document}







